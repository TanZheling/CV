%%%%%%%%%%%%%%%%%%%%%%%%%%%%%%%%%%%%%%%%%
% Medium Length Professional CV
% LaTeX Template
% Version 2.0 (8/5/13)
%
% This template has been downloaded from:
% http://www.LaTeXTemplates.com
%
% Original author:
% Trey Hunner (http://www.treyhunner.com/)
%
% Important note:
% This template requires the resume.cls file to be in the same directory as the
% .tex file. The resume.cls file provides the resume style used for structuring the
% document.
%
%%%%%%%%%%%%%%%%%%%%%%%%%%%%%%%%%%%%%%%%%

%----------------------------------------------------------------------------------------
%	PACKAGES AND OTHER DOCUMENT CONFIGURATIONS
%----------------------------------------------------------------------------------------

\documentclass{resume} % Use the custom resume.cls style

\usepackage[left=0.5in,top=0.6in,right=0.5in,bottom=0.6in]{geometry} % Document margins
\usepackage{graphicx}
\usepackage{wrapfig}
\usepackage{caption}
\newcommand{\tab}[1]{\hspace{.2667\textwidth}\rlap{#1}}
\newcommand{\itab}[1]{\hspace{0em}\rlap{#1}}


\begin{wrapfigure}{r}{0.2\textwidth}\vspace{-25pt} %this figure will be at the right
    \centering
    \includegraphics[width=0.13\textwidth]{photo.jpg}
\end{wrapfigure}
\address{Wuhan University, 430070} % Your address
%\address{123 Pleasant Lane \\ City, State 12345} % Your secondary addess (optional)
\address{ tanzheling@foxmail.com }
\name{Zheling Tan} % Your name


\begin{document}

%----------------------------------------------------------------------------------------
%	EDUCATION SECTION
%----------------------------------------------------------------------------------------

\begin{rSection}{Education}

{\bf Wuhan University} \hfill {\em September 2018 - Present} 
\\ 4th Year M.Sc, Computer Science \hfill { Professional GPA: 3.91/4}
\\  Hongyi Honor College
\\



\end{rSection}
%----------------------------------------------------------------------------------------
%	TECHNICAL STRENGTHS SECTION
%----------------------------------------------------------------------------------------



%----------------------------------------------------------------------------------------
%	WORK EXPERIENCE SECTION
%----------------------------------------------------------------------------------------

\begin{rSection}{Research Experience}

\begin{rSubsection}{Wuhan University}{Wuhan, China}
{NIS\&P Lab}
{Sep 2019-June 2021}
{Group led by Prof.Qian Wang}
 \item Studying papers of related areas and constructing neural networks to implement popular techniques such
as DeepFakes, Faceswap and Faceshifter.
 \item Studying to apply the feature extraction theory of Multi-level Attributes Encoder proposed in
Faceshifter to Faceswap, a face swapping technique that can be applied to videos.
\end{rSubsection}

\end{rSection}

\begin{rSection}{Project Experience}

\begin{rSubsection}{Innovation and Entrepreneurship Trainning of National Undergraduate}{\emph{Apr 2020-Present}}
{\textbf{Guardian}, Group leader}
{}
{}
 \item Constructing and training violence detection model. Equipping Raspberry Pi with violence detection
models and facial recognition models, which made the functions demonstrable.
\end{rSubsection}

\begin{rSubsection}{Smart Tunnel-- tunnel health management system}{\emph{Feb 2021-June 2021}}
{}
{}
{}
 \item React+SpringBoot+MySQL
 \item Leading the development of back-end, which consist of SpringBoot development, database design,
server deployment and the study of prediction algorithm.
\item Second prize in National Computer Design Competition for College Students.
\end{rSubsection}

\begin{rSubsection}{Meritorious Winner in COMAP’s Mathematical Contest in Modeling}{\emph{Feb 2020}}
{”What A Waste! Estimation of plastic waste Reduction”}
{}
{}
 \item Develop model to estimate the maximum levels of plastic waste that can safely be mitigated without further environmental damage;
 \item Use model to set a target for the minimal achievable level of global waste of disposable plastic products.
\end{rSubsection}

\end{rSection}

\begin{rSection}{CORE COURSES}
\item Advanced Mathematics I(89); Advanced Mathematics II(94); Linear algebra (91); Probability theory \&
mathematical statistics (99); Digital logic and digital circuits (95); Fundament of Computer System I (92);
Fundament of Computer System II (90); The design and analysis of algorithm (91); Data Structure (90);
Database System(98); Principles of Operating System (93); Advanced Language Programming(C/C++)
(90);
\end{rSection}


%	EXAMPLE SECTION
%----------------------------------------------------------------------------------------


\end{document}
